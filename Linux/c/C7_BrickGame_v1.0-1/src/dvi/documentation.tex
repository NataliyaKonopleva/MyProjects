\documentclass[12pt, letterpaper]{article}
\usepackage[utf8]{inputenc}

\title{Tetris documentation.}
\author{N.Konopleva}
\date{October, 2024}

\begin{document}
\maketitle

\textbf{Welcome to Tetris!}

The program develop in C language of the C11 standard using the gcc compiler.


Add support for all buttons provided on the physical console for control: 
\begin{itemize}
    \item Start game,
    \item Pause,
    \item End game,
    \item Left arrow — move the piece to the left,
    \item Right arrow — move piece to the right,
    \item Down arrow — piece falls,
    \item Up arrow is not used in this game,
    \item Action (piece rotation).
\end{itemize}


Add the following mechanics to the game 
Scoring;
Store maximum score. 
This information must be passed and displayed by the user interface in the sidebar. The maximum score must be stored in a file or an embedded
DBMS and saved between program runs. 
The maximum score must be changed during the game if the user exceeds the current maximum score. 

Points are accumulated as follows: 

\begin{itemize}
    \item  1 row is 100 points;
    \item  2 rows is 300 points;
    \item  3 rows is 700 points;
    \item  4 rows is 1500 points. 
\end{itemize}

Add level mechanics to the game. Each time a player earns 600 points, the level increases by 1. Increasing the level increases the speed at which
the pieces move. The maximum number of levels is 10


\textit{Have a nice game!}

\end{document}